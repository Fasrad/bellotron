\newcommand{\specname}{bellofy}
\newcommand{\status}{Beta}
\newcommand{\ecn}{NA}
\newcommand{\revdate}{120128}
\newcommand{\rev}{000}

\documentclass[dvips,12pt]{article}
\renewcommand{\contentsname}{3. Index} 

\usepackage{amsmath}
%\usepackage{program}
\usepackage{a4,color,graphics,palatino,fancyhdr}
\usepackage{lastpage}
\usepackage{fancyhdr}
\usepackage{changepage}% http://ctan.org/pkg/changepage
\usepackage{graphicx}
\usepackage{float}

\floatstyle{ruled}
\newfloat{program}{thp}{lop}
\floatname{program}{Source Code}

\setlength{\headheight}{15pt}

\setcounter{secnumdepth}{1}
\setcounter{tocdepth}{1}

\lhead{\specname}
\rhead{rev.\  \rev} 
\chead{\revdate}
\cfoot{\footnotesize Page\ \thepage\ of \pageref{LastPage}}
\pagestyle{fancy}

\title{bellofy}

\author{chaz}

\begin{document}

\frenchspacing

\section{Purpose}
Documentation for bellofy program

\tableofcontents
\listoffigures


\section{What is a bellows}

Google it. It's camera thing. You make them out of fabric or paper or 
leather or something. It's really hard to get them straight and to the 
right dimension. 

I used a CAD program to draw my first bellows pattern, but it was a waste
because it was only valid for the particular bellows I needed then. Bellotron
is designed to generate patterns for arbitrary bellows. 

\section{Pluralization of "bellows"}

It seems to be something of a mass noun; I hear both "a bellows" and 
"the bellows" and just "bellows". I rarely see "a bellow".

\section{Patterns}

The camera bellows pattern I'm using is a hybrid of several techniques
I saw online and what I found to work from making the bellows myself.

In general bellows can be either square in aspect ratio (such as for a square
camera format like 6x6) or non-square (such as for a camera that is wider 
than tall, like a panoramic camera format). Besides that, the bellows may
be tapered from the rear standard of the camera (larger typically) to the
front standard of the camera. Furthermore, we may also wish the pleats 
of the bellows to taper in size from the rear
to the front of the the camera. And the aspect ratio of the bellows can 
even change between the front and rear of the camera...the front may be
square and the rear may be non-square! So you can see how complicated 
it can be to make one and get
it to fit your camera once it's folded. In the past, I have made the 
bellows first, then built the camera around it, but this doesn't work
for making replacements. 

Bellotron is intended to handle all the variations described above. It is
intended for cameras, so it does
not handle non-rectangular bellows (no octagon or triangular bellows)
and it cannot handle pleats that vary in size nonlinearly from the rear
to the front of the camera (some cameras have even-size big pleats for the back half
of the bellows, then the pleats start tapering down from the middle of the bellows 
to the front).

Given this theoretical bellows definition, The essential parameters are the (folded) 
inner and outer widths, the inner and outer height for the front and rear standards, and 
the desired length (when fully extended). Defined this way, the pleat-size is automatically
determined by these constraints and pleat-size linearly changes between the front and 
back standards. 

\section{Technical details}

The patterns are written in PostScript which may be directly printed on most printers.


\centering
\vspace{2cm}
END OF DATA
\appendix

\end{document}

